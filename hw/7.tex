\section{Краевые задачи для уравнения Пуасонна в круге и кольце, шаре и шаровом слое}
\subsection{ 16.1(3) }
$u|_{r=1}= \cos^{4} \varphi$ \\
$\Delta u = 0, \: |r| < 1$ \\
\begin{gather*}
  u_{r r} + \frac{1}{r}u_{r} + \frac{1}{r^{2}}u_{\varphi \varphi} = 0 \\
  \cos^{4} \varphi = \frac{\cos 4 \varphi}{8} + \frac{\cos 2 \varphi}{2} + \frac{3}{8} \\
  u = A + Br^{2}\cos 2 \varphi + Cr^{2}\sin 2 \varphi + Dr^{4} \cos 4 \varphi + E r^{4}\sin 4 \varphi \\
  u|_{r=1} =  A + B\cos 2 \varphi + C\sin 2 \varphi + D \cos 4 \varphi + E \sin 4 \varphi \Rightarrow \\
  A = \frac{3}{8} \quad B = \frac{1}{2} \quad C = 0 \quad D = \frac{1}{8} \quad E = 0 \\
  u = \frac{3}{8}+ \frac{r^{2}}{2}\cos 2 \varphi + \frac{r^{4}}{8}\cos 4 \varphi \\
  \text{if $|r| > 1$:} \\
  u = A + B\ln r + \frac{C}{r^{2}}\cos 2 \varphi + \frac{D}{r^{2}}\sin 2 \varphi + \frac{E}{r^{4}}\sin 4 \varphi 
  + \frac{F}{r^{4}}\sin 4 \varphi \\
  u|_{r=1}= A + C\cos 2 \varphi + D\sin 2 \varphi + E\sin 4 \varphi + F\sin 4 \varphi \\
  A = \frac{3}{8} \quad C = \frac{1}{2} \quad D = 0 \quad E = \frac{1}{8} \quad F = 0 \\
  u = \frac{3}{8} + B\ln r + \frac{1}{2}\cos 2 \varphi + \frac{1}{8} \cos 4 \varphi \quad B = \text{const} \\
\end{gather*} \newpage
\subsection{1} 
$\Delta u = 1, \: r < 2; \quad u_{r}|_{r=2} = 2\sin( \varphi + \frac{\pi}{3}) + 1$ \\
\begin{gather*}
  u_{r r} + \frac{1}{r}u_{r} + \frac{1}{r^{2}}u_{\varphi \varphi} = 1 \\
  u_{\text{частн}} = Cr^{2} \qquad 2A + 2A = 1 \Rightarrow A = \frac{1}{4} \quad u_{\text{частн}} = \frac{r^{2}}{4} \\
  v = u - u_{\text{частн}} \\
  \begin{cases}
    \Delta v = 0\\ v_{r}|_{r=2} = u_{r}|_{r=2} - u_{\text{частн},r}|_{r=2} = \sin \varphi - \sqrt{3} \cos \varphi \\
  \end{cases} \\
    v = A + Br \cos \varphi + Cr \sin \varphi  \\
    v_{r}|_{r=2} = B\cos \varphi + C\sin \varphi \\
    C  = 1 \quad B = \sqrt{3} \\
    v = A + \sqrt{3}r \cos \varphi + r \sin \varphi \\
    u = A + \frac{r^{2}}{4} + \sqrt{3}r\cos \varphi + r\sin \varphi \\
\end{gather*} \\
\subsection{2} 
$\Delta u = \frac{18}{r^{3}}\sin 2 \varphi, \: \frac{1}{2}<r<1;$ \\
$u_{r}|_{r= 1 /2} = 28\sin^{2} \varphi, \quad u_{r}|_{r=1} = 4\cos^{2} \varphi + 5$ \\
\begin{gather*}
  u_{r r} + \frac{u_{r}}{r}+ \frac{u_{\varphi \varphi}}{r^{2}} = \frac{18}{r^{3}} \\
  u_{\text{частн}} = \frac{A \sin 2 \varphi}{r} \\
  \end{gather*}
  \begin{gather*}
  \frac{2A\sin 2 \varphi}{r^{3}} - \frac{A\sin 2 \varphi}{r^{3}} - \frac{4A \sin 2 \varphi}{r^{3}} =
  \frac{18}{r^{3}}\sin 2 \varphi \: \Rightarrow A = -6 \\
  u_{\text{частн}} = - \frac{6\sin 2 \varphi}{r} \\
  v = u - u_{\text{частн}} \\
  v_{r}|_{r = 1 /2} = u_{r}|_{r = 1 /2} - u_{\text{частн},r}|_{r=1 /2} = 28 \sin^{2} \varphi -24
  \sin 2\varphi = 14 - 14\cos 2 \varphi - 24 \sin 2 \varphi \\
v_{r}|_{r = 1} = u_{r}|_{r = 1} - u_{\text{частн},r}|_{r=1} = 4\cos^{2} \varphi - 6 \sin 2 \varphi = 
7 + 2 \cos 2 \varphi - 6 \sin 2 \varphi \\
\end{gather*}
% \item[\text{б)}] \[
%   \begin{cases}
%   u_{tt}= \Delta u +(x^{2}+y^{2})\sin{t} \\ u|_{t=0}=0, \ u_{t}|_{t=0}=(x^{2}+y^{2}+z^{2})^{5 /2}-y^{2}
% \end{cases}
% \]
%   \begin{gather*}
%     u = v+w+p \\
%     1) \begin{cases}
%       v_{tt}= \Delta v + (x^{2}+y^{2}
%     \end{cases}
%   \end{gather*}
% \item[Т4]
%   \begin{enumerate}
% \item[T5] $\Delta u = 4 \frac{(x-y)^{2}}{(x^{2)+y^{2}})^{3}}$
%   \begin{gather*}
%     u_{r r} + \frac{1}{2}u_{r}+ \frac{1}{r^{2}}u_{\varphi \varphi}= 4 \frac{(\cos\varphi-\sin\varphi)^{2}}{r^{4}= \frac{4}{r^{4}}} - \frac{4\sin(2\varphi)}{r^{4}} \\
%     \Delta v = 0 \\
%     v_{r}|_{r=1}= 2\sin \varphi \cos \varphi - \sin 2 \varphi = 2 \\
%     \begin{split}
%     (v_{r}+v)|_{r=2} &= \frac{1}{8} (\sin^{2} \varphi + \cos^{2} \varphi +2\sin \varphi \cos \varphi) + \frac{1}{4} + \frac{1}{4}\ln 2 \varphi \\ 
%                      &- \frac{1}{8} \sin 2 \varphi - \frac{1}{4}  -\frac{1}{4}\ln 2 \sin 2 \varphi = \frac{1}{8} \\
%     \end{split} \\
%     v = A + B \ln r \\
%     v_{r}|_{r=1}=B=2 \\
%     (v_{2}+v)|_{r=2}= \frac{B}{2}+A+B\ln 2 = \frac{1}{8} \\
%     u = \frac{1}{2}r + \frac{1}{r^{2}}\ln r \sin 2 \varphi + (- \frac{7}{8}-2 \ln 2)+2 \ln r \\
%   \end{gather*} \\
% \item[T6]   При каком $\alpha$ $\Delta u = y, \: r < 2$ \\
%   $u_{r}|_{r=2} = \sin^{3} \varphi + \alpha \cos^{2} \varphi$ имеет решение? И найти это решение \\
%   \begin{gather*}
%     \int_{r=2}(\sin^{3} \varphi + \alpha \cos^{2} \varphi  )\,ds = \int_{0}^{2 \pi}
%     (\sin^{3} \varphi + \alpha \cos^{2} \varphi )\,2d \varphi = \\
%     2 \int_{0}^{2\pi} (1-\cos^{3} \varphi)\sin \varphi \,d \varphi = \alpha \int_{0}^{2\pi} (1+\cos 2 \varphi) \,d \varphi = \\
%     2\left.\left[(-\cos \varphi + \frac{\cos^{3} \varphi}{3})\right] \right|_{0}^{2\pi}+
%     \alpha(\varphi + \frac{\sin 2 \varphi}{2})|_{0}^{2\pi} = 2\alpha \pi \\
%     u_{r r} + \frac{1}{r}u_{r}+ \frac{1}{r^{2}}u_{\varphi \varphi} = r \sin \varphi \\
%     u_{\text{частн}} = \alpha r^{3} \sin \varphi \\
%     6 \alpha r \sin \varphi + 3 \alpha r \sin \varphi - \alpha r \sin \varphi = 2 \sin \varphi \\
%     8 \alpha = 1 \Rightarrow \alpha = \frac{1}{8} \\
%     u = \frac{1}{8}r^{3} \sin \varphi + v \\
%     \begin{cases}
%       \Delta v = 0 \\ v_{r}|_{r=2}=\sin 3 \varphi - \frac{3}{2}\sin \varphi = - \frac{3}{4}\sin \varphi - \frac{1}{4} \sin 3 \varphi 
%     \end{cases} \\
%     v = Ar \sin \varphi + Br^{3}\sin 3 \varphi \\
%     v_{r}|_{r=2}= A \sin \varphi + 12 B \sin 3 \varphi = - \frac{3}{4}\sin \varphi - \frac{1}{4}\sin 3 \varphi \\
%     A = - \frac{3}{4} \quad B = - \frac{1}{48} \\
%   \end{gather*}
% \end{enumerate}

% \item[T1] $u_{t}= \frac{1}{4}\Delta u +t^{4}(x+1)$ \\
%   $u|_{t=0}=e^{2z-z^{2}}\sin(x+y)$ \\
%   \begin{gather*}
%     u = v + w(t,z) \cdot p (t,x,y) \\
%     1) \begin{cases}
%       v_{t}= \frac{1}{4}\Delta v +t^{4}(x+1) \\
%       v_{t=0}=0 \\
%     \end{cases} \\
%     v = h(t)(x+1) \\
%     h'(x+1)= \frac{1}{4} h \cdot 0 +t^{4}(x+1) \\
%     \begin{cases}
%       h'=t^{4} \\ h(0) = 0
%     \end{cases} \\
%     h = \frac{t^{5}}{5} \\
%     2) \begin{cases}
%       w_{t}= \frac{1}{4}\Delta w \\ w|_{t=0}=e^{2z-z^{2}}
%     \end{cases} \\
%     w(t,z) = \frac{1}{\sqrt{\pi t}}\int_{-\infty}^{\infty} e^{- \frac{(\xi-z)^{2}}{t}}e^{2\xi-\xi^{2}} \,d\xi \\
%     - \frac{\xi^{2}}{t}+2 \frac{z\xi}{t}- \frac{z^{2}}{t}+2\xi-\xi^{2}=-\xi^{2}( \frac{1}{t}+1)
%     +\xi- \frac{z^{2}}{t} = \\
%     -\left[(\xi \sqrt{ \frac{1}{t}+1})^{2}-2\xi \sqrt{ \frac{1}{t}+1}\cdot \frac{\frac{z}{t}+1}{ \sqrt{ \frac{1}{t}+1}}+ \frac{(\frac{z}{t}+1)^{2}}{ \frac{1}{t}+1}\right]
%     + \frac{(z+t)^{2}}{t(t+1)} - \frac{z^{2}}{t} = -(\xi \sqrt{ \frac{1}{t}+1}- \frac{z+t}{\sqrt{t(t+1)}})^{2}+ \frac{z^{2}+2zt+t^{2}-tz^{2}-z^{2}}{t(t+1)} \\
%     w = \frac{e^{\frac{2z-z^{2}+t}{t+1}}}{\sqrt{\pi t}}\int_{-\infty}^{\infty} e^{-\xi \sqrt{\frac{1}{t}+1}-\frac{z+t}{\sqrt{t(t+1)}}} \,d\xi=
%     =\frac{1}{\sqrt{1+ \frac{1}{t}}}\int_{-\infty}^{\infty} e^{-\xi^{2}} \,d\xi = e^{\frac{2z-z^{2}+t}{\sqrt{t+1}}} \\
%     3) \begin{cases}
%       p_{t=0}= \frac{1}{4}\Delta p \\ p|_{t=0}=\sin(x+y) 
%     \end{cases} \\
%   \end{gather*}
% \item[\text{б)}] \[
%   \begin{cases}
%   u_{tt}= \Delta u +(x^{2}+y^{2})\sin{t} \\ u|_{t=0}=0, \ u_{t}|_{t=0}=(x^{2}+y^{2}+z^{2})^{5 /2}-y^{2}
% \end{cases}
% \]
%   \begin{gather*}
%     u = v+w+p \\
%     1) \begin{cases}
%       v_{tt}= \Delta v + (x^{2}+y^{2}
%     \end{cases}
%   \end{gather*}
% \item[T5] $\Delta u = 4 \frac{(x-y)^{2}}{(x^{2)+y^{2}})^{3}}$
%   \begin{gather*}
%     u_{r r} + \frac{1}{2}u_{r}+ \frac{1}{r^{2}}u_{\varphi \varphi}= 4 \frac{(\cos\varphi-\sin\varphi)^{2}}{r^{4}= \frac{4}{r^{4}}} - \frac{4\sin(2\varphi)}{r^{4}} \\
%     \Delta v = 0 \\
%     v_{r}|_{r=1}= 2\sin \varphi \cos \varphi - \sin 2 \varphi = 2 \\
%     \begin{split}
%     (v_{r}+v)|_{r=2} &= \frac{1}{8} (\sin^{2} \varphi + \cos^{2} \varphi +2\sin \varphi \cos \varphi) + \frac{1}{4} + \frac{1}{4}\ln 2 \varphi \\ 
%                      &- \frac{1}{8} \sin 2 \varphi - \frac{1}{4}  -\frac{1}{4}\ln 2 \sin 2 \varphi = \frac{1}{8} \\
%     \end{split} \\
%     v = A + B \ln r \\
%     v_{r}|_{r=1}=B=2 \\
%     (v_{2}+v)|_{r=2}= \frac{B}{2}+A+B\ln 2 = \frac{1}{8} \\
%     u = \frac{1}{2}r + \frac{1}{r^{2}}\ln r \sin 2 \varphi + (- \frac{7}{8}-2 \ln 2)+2 \ln r \\
%   \end{gather*} \\
% \item[T6]   При каком $\alpha$ $\Delta u = y, \: r < 2$ \\
%   $u_{r}|_{r=2} = \sin^{3} \varphi + \alpha \cos^{2} \varphi$ имеет решение? И найти это решение \\
%   \begin{gather*}
%     \int_{r=2}(\sin^{3} \varphi + \alpha \cos^{2} \varphi  )\,ds = \int_{0}^{2 \pi}
%     (\sin^{3} \varphi + \alpha \cos^{2} \varphi )\,2d \varphi = \\
%     2 \int_{0}^{2\pi} (1-\cos^{3} \varphi)\sin \varphi \,d \varphi = \alpha \int_{0}^{2\pi} (1+\cos 2 \varphi) \,d \varphi = \\
%     2\left.\left[(-\cos \varphi + \frac{\cos^{3} \varphi}{3})\right] \right|_{0}^{2\pi}+
%     \alpha(\varphi + \frac{\sin 2 \varphi}{2})|_{0}^{2\pi} = 2\alpha \pi \\
%     u_{r r} + \frac{1}{r}u_{r}+ \frac{1}{r^{2}}u_{\varphi \varphi} = r \sin \varphi \\
%     u_{\text{частн}} = \alpha r^{3} \sin \varphi \\
%     6 \alpha r \sin \varphi + 3 \alpha r \sin \varphi - \alpha r \sin \varphi = 2 \sin \varphi \\
%     8 \alpha = 1 \Rightarrow \alpha = \frac{1}{8} \\
%     u = \frac{1}{8}r^{3} \sin \varphi + v \\
%     \begin{cases}
%       \Delta v = 0 \\ v_{r}|_{r=2}=\sin 3 \varphi - \frac{3}{2}\sin \varphi = - \frac{3}{4}\sin \varphi - \frac{1}{4} \sin 3 \varphi 
%     \end{cases} \\
%     v = Ar \sin \varphi + Br^{3}\sin 3 \varphi \\
%     v_{r}|_{r=2}= A \sin \varphi + 12 B \sin 3 \varphi = - \frac{3}{4}\sin \varphi - \frac{1}{4}\sin 3 \varphi \\
%     A = - \frac{3}{4} \quad B = - \frac{1}{48} \\
%   \end{gather*}
% \end{enumerate}
